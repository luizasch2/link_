%% Generated by Sphinx.
\def\sphinxdocclass{report}
\documentclass[letterpaper,10pt,english]{sphinxmanual}
\ifdefined\pdfpxdimen
   \let\sphinxpxdimen\pdfpxdimen\else\newdimen\sphinxpxdimen
\fi \sphinxpxdimen=.75bp\relax
\ifdefined\pdfimageresolution
    \pdfimageresolution= \numexpr \dimexpr1in\relax/\sphinxpxdimen\relax
\fi
%% let collapsible pdf bookmarks panel have high depth per default
\PassOptionsToPackage{bookmarksdepth=5}{hyperref}

\PassOptionsToPackage{warn}{textcomp}
\usepackage[utf8]{inputenc}
\ifdefined\DeclareUnicodeCharacter
% support both utf8 and utf8x syntaxes
  \ifdefined\DeclareUnicodeCharacterAsOptional
    \def\sphinxDUC#1{\DeclareUnicodeCharacter{"#1}}
  \else
    \let\sphinxDUC\DeclareUnicodeCharacter
  \fi
  \sphinxDUC{00A0}{\nobreakspace}
  \sphinxDUC{2500}{\sphinxunichar{2500}}
  \sphinxDUC{2502}{\sphinxunichar{2502}}
  \sphinxDUC{2514}{\sphinxunichar{2514}}
  \sphinxDUC{251C}{\sphinxunichar{251C}}
  \sphinxDUC{2572}{\textbackslash}
\fi
\usepackage{cmap}
\usepackage[T1]{fontenc}
\usepackage{amsmath,amssymb,amstext}
\usepackage{babel}



\usepackage{tgtermes}
\usepackage{tgheros}
\renewcommand{\ttdefault}{txtt}



\usepackage[Bjarne]{fncychap}
\usepackage{sphinx}

\fvset{fontsize=auto}
\usepackage{geometry}


% Include hyperref last.
\usepackage{hyperref}
% Fix anchor placement for figures with captions.
\usepackage{hypcap}% it must be loaded after hyperref.
% Set up styles of URL: it should be placed after hyperref.
\urlstyle{same}

\addto\captionsenglish{\renewcommand{\contentsname}{Contents:}}

\usepackage{sphinxmessages}
\setcounter{tocdepth}{1}



\title{A1 \sphinxhyphen{} Queens}
\date{Oct 08, 2022}
\release{1.0.0}
\author{Luiza Schneider, Luana Hatab, Mateus Mourão}
\newcommand{\sphinxlogo}{\vbox{}}
\renewcommand{\releasename}{Release}
\makeindex
\begin{document}

\ifdefined\shorthandoff
  \ifnum\catcode`\=\string=\active\shorthandoff{=}\fi
  \ifnum\catcode`\"=\active\shorthandoff{"}\fi
\fi

\pagestyle{empty}
\sphinxmaketitle
\pagestyle{plain}
\sphinxtableofcontents
\pagestyle{normal}
\phantomsection\label{\detokenize{index::doc}}

\index{module@\spxentry{module}!functions@\spxentry{functions}}\index{functions@\spxentry{functions}!module@\spxentry{module}}\index{duracao\_album() (in module functions)@\spxentry{duracao\_album()}\spxextra{in module functions}}

\begin{fulllineitems}
\phantomsection\label{\detokenize{index:functions.duracao_album}}
\pysigstartsignatures
\pysiglinewithargsret{\sphinxcode{\sphinxupquote{functions.}}\sphinxbfcode{\sphinxupquote{duracao\_album}}}{\emph{\DUrole{n}{df}\DUrole{p}{:}\DUrole{w}{  }\DUrole{n}{DataFrame}}}{{ $\rightarrow$ Series}}
\pysigstopsignatures
\sphinxAtStartPar
Retorna uma série com o nome das músicas e suas durações em ordem decrescente
\begin{quote}\begin{description}
\sphinxlineitem{Parameters}
\sphinxAtStartPar
\sphinxstyleliteralstrong{\sphinxupquote{df}} (\sphinxstyleliteralemphasis{\sphinxupquote{pd.DataFrame}}) \textendash{} dataframe que possui como um dos índices os nomes das músicas e uma de suas colunas é a duração da música

\sphinxlineitem{Returns}
\sphinxAtStartPar
série com o nome das músicas e suas durações em ordem decrescente

\sphinxlineitem{Return type}
\sphinxAtStartPar
pd.Series

\end{description}\end{quote}

\end{fulllineitems}

\index{maiores\_menores\_idx() (in module functions)@\spxentry{maiores\_menores\_idx()}\spxextra{in module functions}}

\begin{fulllineitems}
\phantomsection\label{\detokenize{index:functions.maiores_menores_idx}}
\pysigstartsignatures
\pysiglinewithargsret{\sphinxcode{\sphinxupquote{functions.}}\sphinxbfcode{\sphinxupquote{maiores\_menores\_idx}}}{\emph{\DUrole{n}{df}\DUrole{p}{:}\DUrole{w}{  }\DUrole{n}{DataFrame}}, \emph{\DUrole{n}{indice\_idx}\DUrole{p}{:}\DUrole{w}{  }\DUrole{n}{str}}, \emph{\DUrole{n}{grupo\_idx}\DUrole{p}{:}\DUrole{w}{  }\DUrole{n}{str}}, \emph{\DUrole{n}{coluna}\DUrole{p}{:}\DUrole{w}{  }\DUrole{n}{str}}, \emph{\DUrole{n}{path}\DUrole{p}{:}\DUrole{w}{  }\DUrole{n}{str}}, \emph{\DUrole{n}{opcao}\DUrole{p}{:}\DUrole{w}{  }\DUrole{n}{int}}}{{ $\rightarrow$ DataFrame}}
\pysigstopsignatures
\sphinxAtStartPar
Monta um DataFrame com os elementos de maiores e menores valores por grupo e salva suas visualizações no path
\begin{quote}\begin{description}
\sphinxlineitem{Parameters}\begin{itemize}
\item {} 
\sphinxAtStartPar
\sphinxstyleliteralstrong{\sphinxupquote{df}} (\sphinxstyleliteralemphasis{\sphinxupquote{pd.DataFrame}}) \textendash{} dataframe com indices individuais e multi\sphinxhyphen{}index

\item {} 
\sphinxAtStartPar
\sphinxstyleliteralstrong{\sphinxupquote{indice\_idx}} (\sphinxstyleliteralemphasis{\sphinxupquote{str}}) \textendash{} nome do indice individual de cada elemento

\item {} 
\sphinxAtStartPar
\sphinxstyleliteralstrong{\sphinxupquote{grupo\_idx}} (\sphinxstyleliteralemphasis{\sphinxupquote{str}}) \textendash{} indice coletivo dos elementos

\item {} 
\sphinxAtStartPar
\sphinxstyleliteralstrong{\sphinxupquote{coluna}} (\sphinxstyleliteralemphasis{\sphinxupquote{str}}) \textendash{} nome da coluna a ser analisada

\item {} 
\sphinxAtStartPar
\sphinxstyleliteralstrong{\sphinxupquote{path}} (\sphinxstyleliteralemphasis{\sphinxupquote{str}}) \textendash{} caminho da pasta aonde a img deve ser salva

\item {} 
\sphinxAtStartPar
\sphinxstyleliteralstrong{\sphinxupquote{opcao}} (\sphinxstyleliteralemphasis{\sphinxupquote{int}}) \textendash{} 0 para retornar os menores elementos e 1 para os maiores

\end{itemize}

\sphinxlineitem{Returns}
\sphinxAtStartPar
retorna um df com os elementos de maiores e menores valores por grupo e salva suas visualizações no path

\sphinxlineitem{Return type}
\sphinxAtStartPar
pd.DataFrame

\end{description}\end{quote}

\end{fulllineitems}

\index{nouns() (in module functions)@\spxentry{nouns()}\spxextra{in module functions}}

\begin{fulllineitems}
\phantomsection\label{\detokenize{index:functions.nouns}}
\pysigstartsignatures
\pysiglinewithargsret{\sphinxcode{\sphinxupquote{functions.}}\sphinxbfcode{\sphinxupquote{nouns}}}{\emph{\DUrole{n}{series}\DUrole{p}{:}\DUrole{w}{  }\DUrole{n}{Series}}}{{ $\rightarrow$ Series}}
\pysigstopsignatures
\sphinxAtStartPar
Cria série com todos substantivos presentes nos elementos da série passada como parâmetro
\begin{quote}\begin{description}
\sphinxlineitem{Parameters}
\sphinxAtStartPar
\sphinxstyleliteralstrong{\sphinxupquote{series}} (\sphinxstyleliteralemphasis{\sphinxupquote{pd.Series}}) \textendash{} série que terá seus elementos analisados

\sphinxlineitem{Returns}
\sphinxAtStartPar
série apenas com os substantivos presentes na série passada como parâmetro

\sphinxlineitem{Return type}
\sphinxAtStartPar
pd.Series

\end{description}\end{quote}

\end{fulllineitems}

\index{palavras\_duracao() (in module functions)@\spxentry{palavras\_duracao()}\spxextra{in module functions}}

\begin{fulllineitems}
\phantomsection\label{\detokenize{index:functions.palavras_duracao}}
\pysigstartsignatures
\pysiglinewithargsret{\sphinxcode{\sphinxupquote{functions.}}\sphinxbfcode{\sphinxupquote{palavras\_duracao}}}{\emph{\DUrole{n}{df}\DUrole{p}{:}\DUrole{w}{  }\DUrole{n}{DataFrame}}, \emph{\DUrole{n}{lyrics}\DUrole{p}{:}\DUrole{w}{  }\DUrole{n}{Series}}, \emph{\DUrole{n}{albuns}\DUrole{p}{:}\DUrole{w}{  }\DUrole{n}{Series}}}{{ $\rightarrow$ bool}}
\pysigstopsignatures
\sphinxAtStartPar
Verifica se a quantidade de palavras esta relacionada com o tempo da música
\begin{quote}\begin{description}
\sphinxlineitem{Parameters}\begin{itemize}
\item {} 
\sphinxAtStartPar
\sphinxstyleliteralstrong{\sphinxupquote{df}} (\sphinxstyleliteralemphasis{\sphinxupquote{pd.DataFrame}}) \textendash{} dataframe com todas informações

\item {} 
\sphinxAtStartPar
\sphinxstyleliteralstrong{\sphinxupquote{lyrics}} (\sphinxstyleliteralemphasis{\sphinxupquote{pd.Series}}) \textendash{} série com as letras das músicas

\item {} 
\sphinxAtStartPar
\sphinxstyleliteralstrong{\sphinxupquote{albuns}} (\sphinxstyleliteralemphasis{\sphinxupquote{pd.Series}}) \textendash{} serie com o nome dos álbuns

\end{itemize}

\sphinxlineitem{Returns}
\sphinxAtStartPar
retorna verdadeiro caso a quantidade de palavras esteja relacionada com a duração e falso caso não esteja

\sphinxlineitem{Return type}
\sphinxAtStartPar
bool

\end{description}\end{quote}

\end{fulllineitems}

\index{theme() (in module functions)@\spxentry{theme()}\spxextra{in module functions}}

\begin{fulllineitems}
\phantomsection\label{\detokenize{index:functions.theme}}
\pysigstartsignatures
\pysiglinewithargsret{\sphinxcode{\sphinxupquote{functions.}}\sphinxbfcode{\sphinxupquote{theme}}}{\emph{\DUrole{n}{series1}\DUrole{p}{:}\DUrole{w}{  }\DUrole{n}{Series}}, \emph{\DUrole{n}{series2}\DUrole{p}{:}\DUrole{w}{  }\DUrole{n}{Series}}}{{ $\rightarrow$ Series}}
\pysigstopsignatures
\sphinxAtStartPar
Checa se os substantivos de series1 estão na series2
\begin{quote}\begin{description}
\sphinxlineitem{Parameters}\begin{itemize}
\item {} 
\sphinxAtStartPar
\sphinxstyleliteralstrong{\sphinxupquote{series1}} (\sphinxstyleliteralemphasis{\sphinxupquote{pd.Series}}) \textendash{} série pandas que origina os temas (substantivos)

\item {} 
\sphinxAtStartPar
\sphinxstyleliteralstrong{\sphinxupquote{series2}} (\sphinxstyleliteralemphasis{\sphinxupquote{pd.Series}}) \textendash{} série pandas onde os temas(substantivos) serão procurados

\end{itemize}

\sphinxlineitem{Returns}
\sphinxAtStartPar
série pandas com os temas (substantivos) da series1 presentes na series2, de acordo com a frequencia

\sphinxlineitem{Return type}
\sphinxAtStartPar
pd.Series

\end{description}\end{quote}

\end{fulllineitems}

\index{wordcloud() (in module functions)@\spxentry{wordcloud()}\spxextra{in module functions}}

\begin{fulllineitems}
\phantomsection\label{\detokenize{index:functions.wordcloud}}
\pysigstartsignatures
\pysiglinewithargsret{\sphinxcode{\sphinxupquote{functions.}}\sphinxbfcode{\sphinxupquote{wordcloud}}}{\emph{\DUrole{n}{series}\DUrole{p}{:}\DUrole{w}{  }\DUrole{n}{Series}}, \emph{\DUrole{n}{file}\DUrole{p}{:}\DUrole{w}{  }\DUrole{n}{str}}}{}
\pysigstopsignatures
\sphinxAtStartPar
Cria wordcloud de série
\begin{quote}\begin{description}
\sphinxlineitem{Parameters}\begin{itemize}
\item {} 
\sphinxAtStartPar
\sphinxstyleliteralstrong{\sphinxupquote{series}} (\sphinxstyleliteralemphasis{\sphinxupquote{pd.Series}}) \textendash{} série pandas cujas palavras geraram o wordcloud

\item {} 
\sphinxAtStartPar
\sphinxstyleliteralstrong{\sphinxupquote{file}} (\sphinxstyleliteralemphasis{\sphinxupquote{str}}) \textendash{} diretrizes para salvar o wordcloud

\end{itemize}

\end{description}\end{quote}

\end{fulllineitems}

\index{words() (in module functions)@\spxentry{words()}\spxextra{in module functions}}

\begin{fulllineitems}
\phantomsection\label{\detokenize{index:functions.words}}
\pysigstartsignatures
\pysiglinewithargsret{\sphinxcode{\sphinxupquote{functions.}}\sphinxbfcode{\sphinxupquote{words}}}{\emph{\DUrole{n}{series}\DUrole{p}{:}\DUrole{w}{  }\DUrole{n}{Series}}}{{ $\rightarrow$ Series}}
\pysigstopsignatures
\sphinxAtStartPar
Cria série pandas com todas as palavras de series
\begin{quote}\begin{description}
\sphinxlineitem{Parameters}
\sphinxAtStartPar
\sphinxstyleliteralstrong{\sphinxupquote{series}} (\sphinxstyleliteralemphasis{\sphinxupquote{pd.Series}}) \textendash{} série cujas palavras serão retornadas como elementos de uma nova série

\sphinxlineitem{Returns}
\sphinxAtStartPar
série com todas as palavras presentes em “series” passado como parâmetro

\sphinxlineitem{Return type}
\sphinxAtStartPar
pd.Series

\end{description}\end{quote}

\end{fulllineitems}

\index{words\_avg() (in module functions)@\spxentry{words\_avg()}\spxextra{in module functions}}

\begin{fulllineitems}
\phantomsection\label{\detokenize{index:functions.words_avg}}
\pysigstartsignatures
\pysiglinewithargsret{\sphinxcode{\sphinxupquote{functions.}}\sphinxbfcode{\sphinxupquote{words\_avg}}}{\emph{\DUrole{n}{series}\DUrole{p}{:}\DUrole{w}{  }\DUrole{n}{Series}}}{{ $\rightarrow$ float}}
\pysigstopsignatures
\sphinxAtStartPar
Calcula média de palavras entre os elementos de uma série
\begin{quote}\begin{description}
\sphinxlineitem{Parameters}
\sphinxAtStartPar
\sphinxstyleliteralstrong{\sphinxupquote{series}} (\sphinxstyleliteralemphasis{\sphinxupquote{pd.Series}}) \textendash{} série com as letras das músicas

\sphinxlineitem{Returns}
\sphinxAtStartPar
média de palavras por música

\sphinxlineitem{Return type}
\sphinxAtStartPar
float

\end{description}\end{quote}

\end{fulllineitems}

\index{words\_freq() (in module functions)@\spxentry{words\_freq()}\spxextra{in module functions}}

\begin{fulllineitems}
\phantomsection\label{\detokenize{index:functions.words_freq}}
\pysigstartsignatures
\pysiglinewithargsret{\sphinxcode{\sphinxupquote{functions.}}\sphinxbfcode{\sphinxupquote{words\_freq}}}{\emph{\DUrole{n}{df}\DUrole{p}{:}\DUrole{w}{  }\DUrole{n}{DataFrame}}, \emph{\DUrole{n}{indice}\DUrole{p}{:}\DUrole{w}{  }\DUrole{n}{str}}, \emph{\DUrole{n}{coluna}\DUrole{p}{:}\DUrole{w}{  }\DUrole{n}{str}}}{{ $\rightarrow$ DataFrame}}
\pysigstopsignatures
\sphinxAtStartPar
função retorna um dataframe com as frequencias das palavras mais freqentes da coluna por índice
\begin{quote}\begin{description}
\sphinxlineitem{Parameters}\begin{itemize}
\item {} 
\sphinxAtStartPar
\sphinxstyleliteralstrong{\sphinxupquote{df}} (\sphinxstyleliteralemphasis{\sphinxupquote{pd.DataFrame}}) \textendash{} dataframe com todas informações

\item {} 
\sphinxAtStartPar
\sphinxstyleliteralstrong{\sphinxupquote{indice}} (\sphinxstyleliteralemphasis{\sphinxupquote{str}}) \textendash{} nome do indice pelo qual as frequenciais devem ser agrupadas

\item {} 
\sphinxAtStartPar
\sphinxstyleliteralstrong{\sphinxupquote{coluna}} (\sphinxstyleliteralemphasis{\sphinxupquote{str}}) \textendash{} nome da coluna que contem as strings com as palavras a serem analisadas

\end{itemize}

\sphinxlineitem{Returns}
\sphinxAtStartPar
dataframe com as frequencias das palavras mais frequentes da coluna por indice

\sphinxlineitem{Return type}
\sphinxAtStartPar
pd.DataFrame

\end{description}\end{quote}

\end{fulllineitems}

\index{words\_n\_stopwords() (in module functions)@\spxentry{words\_n\_stopwords()}\spxextra{in module functions}}

\begin{fulllineitems}
\phantomsection\label{\detokenize{index:functions.words_n_stopwords}}
\pysigstartsignatures
\pysiglinewithargsret{\sphinxcode{\sphinxupquote{functions.}}\sphinxbfcode{\sphinxupquote{words\_n\_stopwords}}}{\emph{\DUrole{n}{series}\DUrole{p}{:}\DUrole{w}{  }\DUrole{n}{Series}}}{{ $\rightarrow$ Series}}
\pysigstopsignatures
\sphinxAtStartPar
Cria série pandas com as palavras de series que não são stopwords (i.e. pronomes e artigos)
\begin{quote}\begin{description}
\sphinxlineitem{Parameters}
\sphinxAtStartPar
\sphinxstyleliteralstrong{\sphinxupquote{series}} (\sphinxstyleliteralemphasis{\sphinxupquote{pd.Index}}) \textendash{} série cujas palavras serão filtradas e retornadas como elementos de uma nova série

\sphinxlineitem{Returns}
\sphinxAtStartPar
série com as palavras que não são stopwords

\sphinxlineitem{Return type}
\sphinxAtStartPar
pd.Series

\end{description}\end{quote}

\end{fulllineitems}



\chapter{Indices and tables}
\label{\detokenize{index:indices-and-tables}}\begin{itemize}
\item {} 
\sphinxAtStartPar
\DUrole{xref,std,std-ref}{genindex}

\item {} 
\sphinxAtStartPar
\DUrole{xref,std,std-ref}{modindex}

\item {} 
\sphinxAtStartPar
\DUrole{xref,std,std-ref}{search}

\end{itemize}


\renewcommand{\indexname}{Python Module Index}
\begin{sphinxtheindex}
\let\bigletter\sphinxstyleindexlettergroup
\bigletter{f}
\item\relax\sphinxstyleindexentry{functions}\sphinxstyleindexpageref{index:\detokenize{module-functions}}
\end{sphinxtheindex}

\renewcommand{\indexname}{Index}
\printindex
\end{document}